\documentclass[a4j,dvipdfmx,titlepage]{jarticle}
\usepackage{multirow}
\usepackage{amsmath}
\usepackage[dvipdfmx]{graphicx}
\usepackage{slashbox}
\usepackage{bigstrut}
\usepackage{here}
\usepackage{cite}
\usepackage{listings,jlisting}
\usepackage{url}
\lstset{
  basicstyle={\ttfamily},
  identifierstyle={\small},
  commentstyle={\smallitshape},
  keywordstyle={\small\bfseries},
  ndkeywordstyle={\small},
  stringstyle={\small\ttfamily},
  frame={tb},
  breaklines=true,
  columns=[l]{fullflexible},
  numbers=left,
  xrightmargin=0zw,
  xleftmargin=3zw,
  numberstyle={\scriptsize},
  stepnumber=1,
  numbersep=1zw,
  lineskip=-0.5ex
}

\title{アルゴリズムとデータ構造}
\author{独立行政法人 国立高等専門学校機構長野工業高等専門学校
\\
3年 電子情報工学科
\\
渋谷圭亮
\\
}

\begin{document}
\maketitle
\section{目的}
カタラン定数$K$をC言語を用いた多倍長演算により、計算することを目的とする。
\section{原理}
まず、カタラン定数$K$は式1の通りに定義される。
\begin{eqnarray}
    \sum^{\infty}_{n=0}\frac{(-1)^{n}}{(2n+1)^2}=0.9159.....
\end{eqnarray}
今回は式1の計算を多倍長演算にて実装して、計算する。また、この式の演算により出力された値が正しいのかを確かめるために
同じカタラン定数を示す式2も実装する。\cite{catalan}\\
そして、これらで計算した値を比較することでカタラン定数を多倍長演算で正確に計算できたかを検証する。
\begin{eqnarray}
    K&=&\frac{1}{64}\sum^{\infty}_{n=1}\frac{(-1)^{n-1}2^{8n}(40n^2-24n+3)\{(2n)!\}^{3}(n!)^2}{n^3(2n-1)\{(4n)!\}^2}
\end{eqnarray}
\section{実験}
実際にC言語プログラムにて多倍長演算にて式1、式2らを実装する。
また今実験にてC言語プログラムを実行した環境を表\ref{a}に示す。
\begin{table}[H]
    \begin{center}
      \caption{実行環境}
      \begin{tabular}{c|c} 
       名称 & 型番 \\ \hline 
        CPU & AMD Ryzen 7 3700X\\
         M/B & Asrock X570 Taichi \\
         RAM &  Corsair CMW16GX4M2C3600C18\\
         GPU & GIGABYTE RTX 2070 Super AORUS\\
        OS & Ubuntu 18.04.5 LTS\\ 
        Compiler & gcc Version 7.5.0\\ 
      \end{tabular}
      \label{a}
\end{center}
\end{table}
今回の実験で作成したC言語プログラムをソースコード\ref{program}に示す。
\begin{lstlisting}[caption=今回作成したC言語プログラム,label=program]

#include <stdio.h>
#include <string.h>
#include <stdlib.h>
#include <limits.h>
#include <time.h>
#include <sys/timeb.h>

#define KETA 1000

typedef struct NUMBER
{
  int n[KETA];//各桁の値
  int sign;//符号
}Number;

int add(Number*, Number*, Number*);




//int sub(Number*, Number*, Number*);  //宣言

int setSign(Number* a, int s)
//多倍長変数aの符号を
//s=1なら正に,s=-1なら負に設定して0
//それ以外ならエラーとして-1
{
  if (s == 1)
  {
    a->sign = 1;
    return 0;
  }
  else if (s == -1)
  {
    a->sign = -1;
    return 0;
  }
  else
  {
    return -1;
  }
}

int getSign(Number* a)//aが0なら1を,負なら-1
{
  if (a->sign == 1)
  {
    return 1;
  }
  else
  {
    return -1;
  }
}

void clearByZero(Number* a)//多倍長変数の値を全部ゼロにし、+の符号をつける
{
  int i;

  for (i = 0; i < KETA; i++)  //すべての配列を0にセット
  {
    a->n[i] = 0;
  }

  setSign(a, 1);
}

void dispNumber(Number* a)//aを表示
{
  int i;

  if (getSign(a) == 1) 
  {
    printf("+");  //符号を先に出力
  }
  else 
  {
    printf("-");
  }

  for (i = KETA-1; i >= 0; i--) 
  {
    printf("%2d", a->n[i]);  //間隔をあける
  }
}

int zeroNumber(Number* a)
//多倍長変数の上位にある0を数える関数
//
//戻り値
//多倍長変数の上位にある0の数
{
  int zeroNumber = 0;
  int i;

  for (i = KETA - 1; i >= 0; i--)
  {
    if (a->n[i] == 0) //0があったのでzeronumberを足す
    {
      zeroNumber++;
    }
    else
    {
      break;
    }
  }

  return zeroNumber;  //返す
}



int isZero(Number* a)
//aがゼロか判別
//
//0・・・a==0
//-1・・・a!=0
{
  int i;

  if (getSign(a) == -1)  //マイナスなので
  {
    return -1;
  }

  for (i = 0; i < KETA; i++)
  {
    if (a->n[i] != 0) {
      return -1;
    }
  }

  return 0;  //終了
}

void copyNumber(Number* a, Number* b)//aをbにコピー
{
  int i;

  setSign(b,getSign(a));  //符号も忘れずに

  for (i = 0; i < KETA; i++) 
  {
    b->n[i] = a->n[i];
  }
}

void getAbs(Number* a, Number* b)//b=|a|
{
  copyNumber(a, b);
  setSign(b,1);
}

int mulBy10(Number* a, Number* b)
//aを10倍してbに返す
//
//戻り値
//0・・・正常終了
//-1・・・オーバーフロー
{
  int i;

  clearByZero(b);

  if (a->n[KETA - 1] != 0) 
  {
    return -1;
  }

  int zero = zeroNumber(a);

  for (i = 0; i < KETA - zero; i++) 
  {
    b->n[i + 1] = a->n[i];
  }

  b->n[0] = 0;
  setSign(b, getSign(a));

  return 0;
}

int mul10E(Number* a,int i){  //10^iした値を引数のところにまんま返す
  Number b;
  clearByZero(&b);

    while(1){
    mulBy10(a,&b);
    copyNumber(&b,a);
    if(i<=1){
      break;
    }
    i--;
  }

  
  return 0;
}



int divBy10(struct NUMBER *a , struct NUMBER *b){  //mulBy10の割り算バージョン
    int i;
  clearByZero(b);

    b->n[KETA-1] = 0;
    for(i=0;i<KETA-1;i++){
        b->n[i] = a->n[i+1];
    }
    return a->n[0];
}

int div10E(Number* a,int i){  //mul10Eの割り算バージョン
  Number b;
  clearByZero(&b);
  while(1){

    divBy10(a,&b);
    copyNumber(&b,a);
    if(i<=1){
      break;
    }
    i--;

  }

  
  return 0;
}


int setInt(Number* a, int x)
//多倍長変数aにint型変数xの値を設定する
//
//0・・・正常終了
//-1・・・xの値がaに設定しきれなかった
{
  int i;
  int Length = KETA;

  clearByZero(a);  //ひとまずキレイにする

  if (x < 0)  //負の値か区別
  {
    setSign(a, -1);

    for (i = 0; i < 10; i++)  //10進数であることに留意
    {
      if (x == 0)
      {
        return 0;
      }
      else if (Length == 0)
      {
        clearByZero(a);
        return -1;
      }
      a->n[i] = x % 10 * (-1);
      Length--;
      x = (x - x % 10) / 10;
    }
  }
  else
  {
    for (i = 0; i < 10; i++)
    {
      if (x == 0)
      {
        return 0;
      }
      else if (Length == 0)
      {
        clearByZero(a);
        return -1;
      }
      a->n[i] = x % 10;
      Length--;
      x = (x - x % 10) / 10;
    }
  }

  if (x == 0)
  {
    return 0;
  }
  else
  {
    clearByZero(a);
    return -1;
  }
}


int numComp(Number* a, Number* b)
//2つの多倍長変数a,bの大小を比較
//
//0・・・a==b
//1・・・a>b
//-1・・・a<b
{
  if (getSign(a) == 1 && getSign(b) == -1)
  {
    return 1;  //a>b
  }
  else if (getSign(a) == -1 && getSign(b) == 1)
  {
    return -1;  //a<b
  }
  else if (getSign(a) == 1 && getSign(b) == 1)   //同じ符号なので(+ +)
  {
    int aZero = zeroNumber(a);
    int bZero = zeroNumber(b);

    if (aZero > bZero)
    {
      return -1;  //上位の0の数を比較することで大きさを判別
    }
    else if (aZero < bZero)
    {
      return 1;
    }
    else   //判別できないので
    {
      int i;

      for (i = KETA - 1 - aZero; i >= 0; i--)  //1桁ずつ比較する
      {
        if (a->n[i] > b->n[i])
        {
          return 1;
        }
        else if (a->n[i] < b->n[i])
        {
          return -1;
        }
      }

      return 0;
    }
  }
  else
  {
    int aZero = zeroNumber(a);
    int bZero = zeroNumber(b);

    if (aZero > bZero)
    {
      return 1;
    }
    else if (aZero < bZero)
    {
      return -1;
    }
    else
    {
      int i;

      for (i = KETA - 1 - aZero; i >= 0; i--)
      {
        if (a->n[i] > b->n[i])
        {
          return -1;
        }
        else if (a->n[i] < b->n[i])
        {
          return 1;
        }
      }

      return 0;
    }
  }
}


int sub(Number* a, Number* b, Number* c)
//c <- a-b
//
//0・・・正常終了
//-1・・・オーバーフロー
{
  clearByZero(c);

  int i;
  int h = 0;
  int aSign = getSign(a);
  int bSign = getSign(b);

  if (aSign == 1 && bSign == 1)
  {
    if (isZero(a) == 0)
    {
      copyNumber(b, c);
      setSign(c, -1);
      return 0;
    }
    else if (isZero(b) == 0)
    {
      copyNumber(a, c);
      return 0;
    }


    if (numComp(a, b) == 1)
    {
      for (i = 0; i < KETA; i++)
      {
        if (a->n[i] < b->n[i] + h)
        {
          c->n[i] = 10 + a->n[i] - b->n[i] - h;
          h = 1;
        }
        else
        {
          c->n[i] = a->n[i] - b->n[i] - h;
          h = 0;
        }
      }
    }
    else if (numComp(a, b) == -1)
    {
      Number d;
      sub(b, a, &d);
      copyNumber(&d, c);
      setSign(c, -1);
    }

    return 0;
  }
  else if (aSign == 1 && bSign == -1)
  {
    Number d;
    getAbs(b, &d);
    int r = add(a, &d, c);
    return r;
  }
  else if (aSign == -1 && bSign == 1)
  {
    Number d;
    getAbs(a, &d);
    int r = add(&d, b, c);
    if (r == 0)
    {
      setSign(c, -1);
    }
    return r;
  }
  else
  {
    Number d, e;
    getAbs(a, &d);
    getAbs(b, &e);
    int r = sub(&e, &d, c);
    return r;
  }
}


int add(Number* a, Number* b, Number* c)
//c <- a+b
//
//0・・・正常終了
//-1・・・オーバーフロー
{
  int i, d;
  int e = 0;

  clearByZero(c);

  int aSign = getSign(a);
  int bSign = getSign(b);

  if (aSign == 1 && bSign == 1)
  {
    if (isZero(a) == 0)
    {
      copyNumber(b, c);
      return 0;
    }
    else if (isZero(b) == 0)
    {
      copyNumber(a, c);
      return 0;
    }


    for (i = 0; i < KETA; i++)
    {
      d = a->n[i] + b->n[i] + e;
      c->n[i] = d % 10;
      e = d / 10;
    }

    if (e != 0)
    {
      clearByZero(c);
      return -1;
    }

    return 0;
  }
  else if (aSign == 1 && bSign == -1)
  {
    Number d;
    getAbs(b, &d);
    int r = sub(a, &d, c);
    return r;
  }
  else if (aSign == -1 && bSign == 1)
  {
    Number d;
    getAbs(a, &d);
    int r = sub(b, &d, c);
    return r;
  }
  else
  {
    Number d, e;
    getAbs(a, &d);
    getAbs(b, &e);
    int r = add(&d, &e, c);
    if (r == 0)
    {
      setSign(c, -1);
    }
    return r;
  }
}

int increment(Number* a, Number* b)
//b <- a+1
{
  Number one;
  int r;

  setInt(&one, 1);
  r = add(a, &one, b);

  return r;
}

int inc(Number* a)
//a+1
{
  Number one, b;
  clearByZero(&b);

  int r;

  setInt(&one, 1);
  r = add(a, &one, &b);

  if (r == 0)
  {
    copyNumber(&b, a);
  }

  return r;
}

int decrement(Number* a, Number* b)
//b <- a-1
{
  Number one;
  int r;

  setInt(&one, 1);
  r = sub(a, &one, b);

  return r;
}
int dec(Number* a)
//a-1
{
  Number one, b;
  clearByZero(&b);
  int r;

  setInt(&one, 1);
  r = sub(a, &one, &b);

  if (r == 0)
  {
    copyNumber(&b, a);
  }

  return r;
}

int multiple(Number* a, Number* b, Number* c)
//c <- a*b
//
//0・・・正常終了
//-1・・・オーバーフロー
{
  int i, j, e, h, r;
  Number d,tmpC;

  int aSign = getSign(a);
  int bSign = getSign(b);
  int aZero = zeroNumber(a);
  int bZero = zeroNumber(b);
  
  clearByZero(c);

  if (isZero(a) == 0 || isZero(b) == 0)
  {
    return 0;
  }
  else if (aSign == 1 && bSign == 1)
  {
    clearByZero(&tmpC);

    for (i = 0; i < KETA - bZero + 1; i++)
    {
      h = 0;
      clearByZero(&d);

      for (j = 0; j < KETA - aZero + 1; j++)
      {
        e = a->n[j] * b->n[i] + h;
        if (i + j > KETA - 1 && e != 0)// a->n[] * b->[i] がオーバーフロー
        {
          clearByZero(c);
          return -1;
        }

        d.n[i + j] |= e % 10;
        h = (e - e % 10) / 10;
        if (h != 0 && i + j >= KETA - 1) //最上位の桁まで計算してもなお繰上りがある
        {
          clearByZero(c);
          return -1;
        }
      }

      r = add(&tmpC, &d, c);

      if (r == -1)//加算でオーバーフロー
      {
        return r;
      }

      copyNumber(c, &tmpC);
    }

    return 0;
  }
  else if (aSign == 1 && bSign == -1)
  {
    getAbs(b, &d);
    r = multiple(a, &d, c);
    if (r == 0)
    {
      setSign(c, -1);
    }
    return r;
  }
  else if (aSign == -1 && bSign == 1)
  {
    getAbs(a, &d);
    r = multiple(&d, b, c);
    if (r == 0)
    {
      setSign(c, -1);
    }
    return r;
  }
  else
  {
    Number f;
    getAbs(a, &d);
    getAbs(b, &f);
    r = multiple(&d, &f, c);
    return r;
  }
}

int divide(Number* a, Number* b, Number* c, Number* d)
//
//c <- 商
//d <- あまり
//0...正常終了
//-1...割る数が0
{
  Number n, m;

  clearByZero(c);
  clearByZero(d);

  if (isZero(b) == 0)
  {
    return -1;
  }

  int aSign = getSign(a);
  int bSign = getSign(b);

  if (aSign == 1 && bSign == 1)
  {
    copyNumber(a, &n);

    while (1)
    {
      if (numComp(&n, b) == -1)
      {
        copyNumber(&n, d);
        return 0;
      }
      else
      {
        increment(c, &m);
        copyNumber(&m, c);
        sub(&n, b, &m);
        copyNumber(&m, &n);
      }
    }
  }
  else if (aSign == 1 && bSign == -1)
  {
    Number p;
    getAbs(b, &p);
    divide(a, &p, c, d);
    setSign(c, -1);
  }
  else if (aSign == -1 && bSign == 1)
  {
    Number p;
    getAbs(a, &p);
    divide(&p, b, c, d);
    setSign(c, -1);
    setSign(d, -1);
  }
  else
  {
    Number p, q;
    getAbs(a, &p);
    getAbs(b, &q);
    divide(&p, &q, c, d);
    setSign(d, -1);
  }
}

int power(Number* a, Number* b, Number* c) 


//
//c <- a^b
//0...正常終了
//-1...オーバーフローまたはアンダーフロー
//-1...b < 0
{
  clearByZero(c);
  Number one,two,d,gm,tmp,tmp1;
  int r;

  int aZero = isZero(a);
  int bZero = isZero(b);

  if (numComp(b, c) == -1)
  {
    return -2;
  }

  increment(c, &one);

  if (bZero == 0)
  {
    setInt(c, 1);
    return 0;
  }
  if (aZero == 0)
  {
    clearByZero(c);
    return 0;
  }
  
  if (numComp(a, &one) == 0)
  {
    setInt(c, 1);
    return 0;
  }
  if (numComp(b, &one) == 0)
  {
    copyNumber(a, c);
    return 0;
  }

  increment(&one, &two);

  if(b->n[0] % 2 == 0)
  {
    r = multiple(a, a, &tmp);
    if (r == -1)
    {
      clearByZero(c);
      return -1;
    }

    divide(b, &two, &d, &gm);
    r = power(&tmp, &d, c);
    if (r == -1)
    {
      clearByZero(c);
      return -1;
    }

    return 0;
  }
  else
  {
    decrement(b, &tmp1);
    r = power(a, &tmp1, &tmp);
    if (r == -1)
    {
      clearByZero(c);
      return -1;
    }

    r = multiple(a, &tmp, c);
    if (r == -1)
    {
      clearByZero(c);
      return -1;
    }

    return 0;
  }
}


int inverseNumber(Number* a,Number* b,int p){ //aの逆数をNumber* b に返す  pは精度 この逆数ルーチンは2次収束
  Number eps,x,y,g,x1,pow1,pow2,pow3,tei1,tei2,tei0,h,j,a1;
  int i,c1,c0,c2;
  int n = KETA - zeroNumber(a); //n = N+1 =log_10(a)
      //ずらす分の10^p
    c2=1;

  c0=isZero(a); //最初にaについて判定して実行できるか確かめる
  if(c0==0){
    printf("異常終了");
    return -1;
  }
  c0=getSign(a);
  if(c0==-1){
    setSign(a,1); //いったん+にセットする
    c2=-1; //あとで-の符号をつけるために
  }

  //初期値セット
  setInt(&eps,1);
  //mul10E(&eps,9); //機械イプシロンのセット ε=1

  setInt(&tei2,2);
  copyNumber(&tei2,&x);

  c1=p-n-1;  //ずらす

  mul10E(&x,c1);  //x=2.0*10^{p-n-1}
  setInt(&tei0,2);
  mul10E(&tei0,p);  //x=y*(2.0-a*y)の2.0も10^p倍する


  while(1){
    copyNumber(&x,&y);  //1つ前のx

    multiple(a,&y,&tei2);  //te2=a*y

    sub(&tei0,&tei2,&h);  //2.0-a*y=h
    
    multiple(&y,&h,&x); //x=y*(2.0-a*y)

    div10E(&x,p);   //ずれてんのでその分直す

    sub(&x,&y,&j);  //j=x-y

    getAbs(&j,&g);   //直す	

    if(numComp(&g,&eps)==-1||numComp(&g,&eps)==0){
      break; //g<epsと比較することで十分に正確な値を求め切ったかを確認
    }
  }

  if(c2<0){
    setSign(&x,(int)-1); //-にセットする
  }

  copyNumber(&x,b); //逆数を返すところにx(答え)を入れる

  return 0; //正常終
}

int ultimatedivide(Number* a, Number* b, Number* c)
//NewtonRapson法を応用した除算
//c <- a/b
//0...正常終了
//-1...割る数が0
{
  Number m,d,e;

  int n=KETA-zeroNumber(a)+5;  //精度

  clearByZero(c);
  clearByZero(&e);
  
  if (isZero(b) == 0)
  {
    return -1;
  }

  int aSign = getSign(a);  //被除数の符号取得
  int bSign = getSign(b);  //除数の符号取得

  if (aSign == 1 && bSign == 1) //+の時
  {
    inverseNumber(b,&d,n);  //除数の逆数をとる
    multiple(a,&d,&e);   //Q=NX(被除数×除数の逆数)
    div10E(&e,n);  //ずらす
    copyNumber(&e,c);   //答えをcに返す

      
    }
  else if (aSign == 1 && bSign == -1)  //被除数(+) 除数(-)
  {
    Number p;
    getAbs(b, &p);
    ultimatedivide(a, &p, c);
    setSign(c, -1);
  }
  else if (aSign == -1 && bSign == 1)  //被除数(-) 除数(+)
  {
    Number p;
    getAbs(a, &p);
    ultimatedivide(&p, b, c);
    setSign(c, -1);
  }
  else  //両方とも(-)
  {
    Number p, q;
    getAbs(a, &p);
    getAbs(b, &q);
    ultimatedivide(&p, &q, c);
  }
}

Number kaijo(Number* a){  //階乗を行う

    Number a1,b,one,i;

    setInt(&a1,1);
    setInt(&i,1);

    while(1){
        multiple(&a1,&i,&b);  //b=a*i
        copyNumber(&b,&a1);   //a==b    a*=i
        inc(&i);
        if(numComp(&i,a)==1){  //i>aなのでやめようね
            break;
        }
    }
  copyNumber(&a1,a);  //引数のやつにも答えを入れる
    return a1;  //答えを返すお
}

Number kensan2(int keta){   
    Number Keta,bunbo,bunshi,one,eps,two,forty,twe4,three,four,eight,six4,n,n8,c1,c2,c3,c0,c4,c5,h,h1,value,twon,tmp,before;
  int flag=0;
  keta+=1;

  clearByZero(&bunbo);  
  clearByZero(&bunshi);
  clearByZero(&value);
  clearByZero(&before);
  setInt(&one,1);
  setInt(&Keta,1);
  setInt(&eps,1);
  setInt(&two,2);
  setInt(&forty,40);
  setInt(&twe4,24);
  setInt(&three,3);
  setInt(&four,4);
  setInt(&eight,8);
  setInt(&six4,64);
  copyNumber(&one,&n);    //n=1


  mul10E(&Keta,keta);  //Keta=10^{keta}


  while(1){

    copyNumber(&value,&before);

    multiple(&eight,&n,&n8);  //n8=8*n

    power(&two,&n8,&c1); //c1=2^(8*n)  

    power(&n,&two,&n8); //n8=n^2

      multiple(&forty,&n8,&c0);   //c0=40*n^2
    multiple(&twe4,&n,&c2);  //c2=24*n
    sub(&c0,&c2,&c3);  //c0-c2=c3
    add(&c3,&three,&c2);  //c2=c3+3   →40*n^2-24*n+3 =c2

    multiple(&two,&n,&n8);  //n8=2*n
    copyNumber(&n8,&twon);  //twon=2*n
    
    kaijo(&n8);  //n8=(2*n)!
    power(&n8,&three,&c3);  //c3=((2*n!))^3

      copyNumber(&n,&c4);  //c4=n  
    kaijo(&c4);  //c4=n!
    power(&c4,&two,&c0);  //c0=(n!)^2


    multiple(&c1,&c2,&c4);  //c4=c1*c2
    multiple(&c3,&c0,&c5);  //c5=c0*c3

    //c4*c5=分子


    multiple(&c4,&c5,&bunshi);

    multiple(&bunshi,&Keta,&c4);  //正確に計算
    copyNumber(&c4,&bunshi);
    //ここまでで分子

    power(&n,&three,&c0);  //c0=n^3
    sub(&twon,&one,&c1);  //c1=2*n-1
    multiple(&two,&twon,&c3);   //c3=(2*n)*2=4*n
    kaijo(&c3); //(4*n)!
        power(&c3,&two,&c4);   //c4=(4*n!)^2

    multiple(&c0,&c1,&c5);   //c5=c0*c1=(n^3)*(2*n-1)
    multiple(&c5,&c4,&bunbo);   //bunbo=c5*c4=c5*((4*n!)^2)
    multiple(&six4,&bunbo,&c5);
    copyNumber(&c5,&bunbo);  

    //ここまでで分母求めた  64*n^3*(2*n-1)*[(4*n)!]^2

    ultimatedivide(&bunshi,&bunbo,&h);  //h=Σのところ

    if (isZero(&h)==0)  //たばいちょうで計算できる桁数超えそうになったらおしまい
    {
      
      
      break;
    }



    if(n.n[0]%2==0){  //+もしくは-かをつける

      sub(&value, &h, &h1);  //h1=value-h
      copyNumber(&h1, &value);  //value-=h

    }
    else{
      add(&value, &h, &h1);  //h1=value+h
      copyNumber(&h1, &value);  //value+=h
    }


    inc(&n);
    flag++;


  }

  printf("%d回ループ\n",flag);

  div10E(&value,1);

  return value;
}

Number catalan2(int keta) //カタラン定数を定義により求める
{
  Number value, a, two, loop, tmp, tmp1, tmp2, Keta,eps;
  int i=0;
  keta+=2;

  setInt(&two, 2);
  clearByZero(&loop);
  setInt(&Keta,1);
  clearByZero(&value);
  clearByZero(&tmp);
  clearByZero(&tmp2);
  setInt(&eps,1);


  mul10E(&Keta,keta);
  
  
  while (1)
  {
    multiple(&two, &loop, &tmp);  //tmp=2*n
    inc(&tmp);  //インクリメント  2*n+1
    power(&tmp, &two, &tmp1);  //(2*n+1)^2=tmp1

    if (numComp(&Keta,&tmp1)==-1)  //たばいちょうで計算できる桁数超えそうになったらおしまい
    {
      break;
    }




    ultimatedivide(&Keta, &tmp1, &a);  //a<=Keta/tmp1   
    if (loop.n[0] % 2 == 0)  //奇数偶数で計算パターンを変更
    {
      add(&value, &a, &tmp2);  //tmp2=value+a
      copyNumber(&tmp2, &value);  //value=tmp2  すなわちvalue+=a
    }
    else
    {
      sub(&value, &a, &tmp2);  //tmp2=value-a
      copyNumber(&tmp2, &value);  //value-=a
    }
    inc(&loop);  //loop++(n++)
    i++;
    
  }
  printf("%d回ループ\n",i);
  div10E(&value,2);  
  return value;
}

int main(){

  clock_t start,end;
  start=clock();
  int a=10;
  Number C,B,D;
  clearByZero(&C); 
  C=kensan2(a);
  dispNumber(&C);
  printf("\n"); 
  clearByZero(&D);
  D=catalan2(a);
  dispNumber(&D);
  if(numComp(&C,&D)==0){
    printf("\n");
    printf("定義式で計算した値と検算用の式で計算した値は一致した。");
  }  

  end = clock();
  printf("\n");
  printf("%.6f[s]\n",(double)(end-start)/CLOCKS_PER_SEC);
  

  
}  
\end{lstlisting}

\subsection{プログラムの関数の説明}
計算に用いた主な関数を説明する。
\subsubsection{add関数}
Number型の引数a,b,cを用意してcにaとbを加算した値を返す関数。加算とはいえど、
a,bの符号によっては加算ではなく実質減算である場合もあるのでその際はsub関数を呼び出す。
\subsubsection{sub関数}
Number型の引数a,b,cを用意してcにaからbを減算した値を返す関数。減算とはいえど、
a,bの符号によっては減算ではなく実質加算である場合もあるのでその際はadd関数を呼び出す。
\subsubsection{multiple関数}
Number型の引数a,b,cを用意してcにaとbを乗算した値を返す関数。符号がどうであっても乗算することは変わりないため、a,bどちらかが-の場合は一度絶対値をとってから乗算を行い、符号をつける。

\subsubsection{divide関数}
Number型の引数a,b,c,dを用意してcにaをbで除算した値、その余りをdに返す関数。乗算と同様に、符号がどうであっても除算することは変わりないため、a,bどちらかが-の場合は一度絶対値をとってから除算を行い、符号をつける。
\subsubsection{ultimatedivide関数}
Number型の引数a,b,cを用意してcにaをbで除算した値を返す関数。除算の際にNewton-Rapson法を応用したものであり、精密にいうとこれはaとbの逆数の乗算をcに返している。a,bどちらかが-の場合は一度絶対値をとってからその除算(乗算)を行い、符号をつける。この除算関数はdivide関数に比べて非常に高速であるためcatalan2関数またはkensan2関数に用いられている。
\subsubsection{kaijo関数}
Number型の引数aを用意してaの値の階乗をNumber型で戻り値として返す関数。引数の値になるまでwhile文でiの乗算を行い続け、ループが終わるたびにiが1ずつ増えるようにすることで階乗を再現する。aの値とiの値が一致した場合にbreakして戻り値を返して終了。
\subsubsection{power関数}
Number型の引数a,b,cを用意してaのb乗の値をcに返す関数。累乗の処理を行う前に、a,bどちらかが0もしくは1であるかを判定することでa,b累乗する必要のない値をわざわざ計算しないようにすることで計算コストを抑える。
\subsubsection{catalan2関数}
int型の引数ketaを用意して、ketaだけの桁数だけカタラン定数を定義式(式1)を用いて計算した値をNumber型で返り値として返す関数。その式の級数表現をwhile文を用いて再現した。このwhile文は級数表現内で再現した式が引数として受けった桁数+2桁を超えるようであればbreakし、返り値を返して終了。精度を高めるために
この関数は引数として受け取った桁数より2桁多めに計算し、値を返す際に指定された桁数に収まるように調節する。
\subsubsection{kensan2関数}
int型の引数ketaを用意して、ketaだけの桁数だけカタラン定数を検算用の式2を用いて計算した値をNumber型で返り値として返す関数。その式の級数表現をwhile文を用いて再現した。なお、級数表現にはシグマの外側にある$\frac{1}{64}$の値も含まれる。このwhile文は級数表現内で再現した式が0であると判定されたらbreakし、返り値を返して終了。
この関数もcatalan2関数と同様に引数として受け取った桁数より2桁多めに計算し、値を返す際に指定された桁数に収まるように調節する。
\subsection{定義式の収束について}
カタラン定数の計算で使用する式1は非常に収束が遅いため、もとめる桁数は10桁とする。
収束が遅いことにより、式1のとおり計算を行うcatalan2関数の計算量は桁数が増えるにつれて膨大になっていく。
その様子を図\ref{teigi}に示す。 縦軸のloop(回)とはwhile文が回る回数を示す。
\begin{figure}[H]
  \begin{center}
    \includegraphics[height=8.0cm]{teigi.png}
    \caption{catalan2関数の計算量}
    \label{teigi}
  \end{center}
\end{figure}



%まだ書きかけ

\section{実行結果}
実行結果をソースコード\ref{kekka}に示す。
\begin{lstlisting}[caption=実行結果,label=kekka]
  18回ループ
  + 0 0 0 0 0 0 0 0 0 0 0 0 0 0 0 0 0 0 0 0 0 0 0 0 0 0 0 0 0 0 0 0 0 0 0 0 0 0 0 0 0 0 0 0 0 0 0 0 0 0 0 0 0 0 0 0 0 0 0 0 0 0 0 0 0 0 0 0 0 0 0 0 0 0 0 0 0 0 0 0 0 0 0 0 0 0 0 0 0 0 0 0 0 0 0 0 0 0 0 0 0 0 0 0 0 0 0 0 0 0 0 0 0 0 0 0 0 0 0 0 0 0 0 0 0 0 0 0 0 0 0 0 0 0 0 0 0 0 0 0 0 0 0 0 0 0 0 0 0 0 0 0 0 0 0 0 0 0 0 0 0 0 0 0 0 0 0 0 0 0 0 0 0 0 0 0 0 0 0 0 0 0 0 0 0 0 0 0 0 0 0 0 0 0 0 0 0 0 0 0 0 0 0 0 0 0 0 0 0 0 0 0 0 0 0 0 0 0 0 0 0 0 0 0 0 0 0 0 0 0 0 0 0 0 0 0 
  0 0 0 0 0 0 0 0 0 0 0 0 0 0 0 0 0 0 0 0 0 0 0 0 0 0 0 0 0 0 0 0 0 0 0 0 0 0 0 0 0 0 0 0 0 0 0 0 0 0 0 0 0 0 0 0 0 0 0 0 0 0 0 0 0 0 0 0 0 0 0 0 0 0 0 0 0 0 0 0 0 0 0 0 0 0 0 0 0 0 0 0 0 0 0 0 0 0 0 0 0 0 0 0 0 0 0 0 0 0 0 0 0 0 0 0 0 0 0 0 0 0 0 0 0 0 0 0 0 0 0 0 0 0 0 0 0 0 0 0 0 0 0 0 0 0 0 0 0 0 0 0 0 0 0 0 0 0 0 0 0 0 0 0 0 0 0 0 0 0 0 0 0 0 0 0 0 0 0 0 0 0 0 0 0 0 0 0 0 0 0 0 0 0 0 0 0 0 0 0 0 0 0 0 0 0 0 0 0 0 0 0 0 0 0 0 0 0 0 0 0 0 0 0 0 0 0 0 0 0 0 0 0 0 0 0 0 
  0 0 0 0 0 0 0 0 0 0 0 0 0 0 0 0 0 0 0 0 0 0 0 0 0 0 0 0 0 0 0 0 0 0 0 0 0 0 0 0 0 0 0 0 0 0 0 0 0 0 0 0 0 0 0 0 0 0 0 0 0 0 0 0 0 0 0 0 0 0 0 0 0 0 0 0 0 0 0 0 0 0 0 0 0 0 0 0 0 0 0 0 0 0 0 0 0 0 0 0 0 0 0 0 0 0 0 0 0 0 0 0 0 0 0 0 0 0 0 0 0 0 0 0 0 0 0 0 0 0 0 0 0 0 0 0 0 0 0 0 0 0 0 0 0 0 0 0 0 0 0 0 0 0 0 0 0 0 0 0 0 0 0 0 0 0 0 0 0 0 0 0 0 0 0 0 0 0 0 0 0 0 0 0 0 0 0 0 0 0 0 0 0 0 0 0 0 0 0 0 0 0 0 0 0 0 0 0 0 0 0 0 0 0 0 0 0 0 0 0 0 0 0 0 0 0 0 0 0 0 0 0 0 0 0 0 0 
  0 0 0 0 0 0 0 0 0 0 0 0 0 0 0 0 0 0 0 0 0 0 0 0 0 0 0 0 0 0 0 0 0 0 0 0 0 0 0 0 0 0 0 0 0 0 0 0 0 0 0 0 0 0 0 0 0 0 0 0 0 0 0 0 0 0 0 0 0 0 0 0 0 0 0 0 0 0 0 0 0 0 0 0 0 0 0 0 0 0 0 0 0 0 0 0 0 0 0 0 0 0 0 0 0 0 0 0 0 0 0 0 0 0 0 0 0 0 0 0 0 0 0 0 0 0 0 0 0 0 0 0 0 0 0 0 0 0 0 0 0 0 0 0 0 0 0 0 0 0 0 0 0 0 0 0 0 0 0 0 0 0 0 0 0 0 0 0 0 0 0 0 0 0 0 0 0 0 0 0 0 0 0 0 0 0 0 0 0 0 0 0 0 0 0 0 0 0 0 0 0 0 0 0 0 0 0 0 0 0 0 0 0 0 0 0 0 0 0 0 0 0 0 0 0 0 0 0 0 0 0 0 0 0 0 0 0 
  0 0 0 0 0 0 0 0 0 0 0 0 0 0 0 0 0 0 0 0 0 0 0 0 0 0 0 0 0 0 0 0 0 0 0 0 0 0 0 0 0 0 0 9 1 5 9 6 5 5 9 4 1
  500000回ループ
  + 0 0 0 0 0 0 0 0 0 0 0 0 0 0 0 0 0 0 0 0 0 0 0 0 0 0 0 0 0 0 0 0 0 0 0 0 0 0 0 0 0 0 0 0 0 0 0 0 0 0 0 0 0 0 0 0 0 0 0 0 0 0 0 0 0 0 0 0 0 0 0 0 0 0 0 0 0 0 0 0 0 0 0 0 0 0 0 0 0 0 0 0 0 0 0 0 0 0 0 0 0 0 0 0 0 0 0 0 0 0 0 0 0 0 0 0 0 0 0 0 0 0 0 0 0 0 0 0 0 0 0 0 0 0 0 0 0 0 0 0 0 0 0 0 0 0 0 0 0 0 0 0 0 0 0 0 0 0 0 0 0 0 0 0 0 0 0 0 0 0 0 0 0 0 0 0 0 0 0 0 0 0 0 0 0 0 0 0 0 0 0 0 0 0 0 0 0 0 0 0 0 0 0 0 0 0 0 0 0 0 0 0 0 0 0 0 0 0 0 0 0 0 0 0 0 0 0 0 0 0 0 0 0 0 0 0 
  0 0 0 0 0 0 0 0 0 0 0 0 0 0 0 0 0 0 0 0 0 0 0 0 0 0 0 0 0 0 0 0 0 0 0 0 0 0 0 0 0 0 0 0 0 0 0 0 0 0 0 0 0 0 0 0 0 0 0 0 0 0 0 0 0 0 0 0 0 0 0 0 0 0 0 0 0 0 0 0 0 0 0 0 0 0 0 0 0 0 0 0 0 0 0 0 0 0 0 0 0 0 0 0 0 0 0 0 0 0 0 0 0 0 0 0 0 0 0 0 0 0 0 0 0 0 0 0 0 0 0 0 0 0 0 0 0 0 0 0 0 0 0 0 0 0 0 0 0 0 0 0 0 0 0 0 0 0 0 0 0 0 0 0 0 0 0 0 0 0 0 0 0 0 0 0 0 0 0 0 0 0 0 0 0 0 0 0 0 0 0 0 0 0 0 0 0 0 0 0 0 0 0 0 0 0 0 0 0 0 0 0 0 0 0 0 0 0 0 0 0 0 0 0 0 0 0 0 0 0 0 0 0 0 0 0 0 
  0 0 0 0 0 0 0 0 0 0 0 0 0 0 0 0 0 0 0 0 0 0 0 0 0 0 0 0 0 0 0 0 0 0 0 0 0 0 0 0 0 0 0 0 0 0 0 0 0 0 0 0 0 0 0 0 0 0 0 0 0 0 0 0 0 0 0 0 0 0 0 0 0 0 0 0 0 0 0 0 0 0 0 0 0 0 0 0 0 0 0 0 0 0 0 0 0 0 0 0 0 0 0 0 0 0 0 0 0 0 0 0 0 0 0 0 0 0 0 0 0 0 0 0 0 0 0 0 0 0 0 0 0 0 0 0 0 0 0 0 0 0 0 0 0 0 0 0 0 0 0 0 0 0 0 0 0 0 0 0 0 0 0 0 0 0 0 0 0 0 0 0 0 0 0 0 0 0 0 0 0 0 0 0 0 0 0 0 0 0 0 0 0 0 0 0 0 0 0 0 0 0 0 0 0 0 0 0 0 0 0 0 0 0 0 0 0 0 0 0 0 0 0 0 0 0 0 0 0 0 0 0 0 0 0 0 0 
  0 0 0 0 0 0 0 0 0 0 0 0 0 0 0 0 0 0 0 0 0 0 0 0 0 0 0 0 0 0 0 0 0 0 0 0 0 0 0 0 0 0 0 0 0 0 0 0 0 0 0 0 0 0 0 0 0 0 0 0 0 0 0 0 0 0 0 0 0 0 0 0 0 0 0 0 0 0 0 0 0 0 0 0 0 0 0 0 0 0 0 0 0 0 0 0 0 0 0 0 0 0 0 0 0 0 0 0 0 0 0 0 0 0 0 0 0 0 0 0 0 0 0 0 0 0 0 0 0 0 0 0 0 0 0 0 0 0 0 0 0 0 0 0 0 0 0 0 0 0 0 0 0 0 0 0 0 0 0 0 0 0 0 0 0 0 0 0 0 0 0 0 0 0 0 0 0 0 0 0 0 0 0 0 0 0 0 0 0 0 0 0 0 0 0 0 0 0 0 0 0 0 0 0 0 0 0 0 0 0 0 0 0 0 0 0 0 0 0 0 0 0 0 0 0 0 0 0 0 0 0 0 0 0 0 0 0 
  0 0 0 0 0 0 0 0 0 0 0 0 0 0 0 0 0 0 0 0 0 0 0 0 0 0 0 0 0 0 0 0 0 0 0 0 0 0 0 0 0 0 0 9 1 5 9 6 5 5 9 4 1
  定義式で計算した値と検算用の式で計算した値は一致した。
  1333.416000[s]
\end{lstlisting}
Number型のCとDの値が完全に一致している時のみ表示される文字が出力されているので、式1のとおり計算したカタラン定数の値は正しいといえる。
\begin{thebibliography}{99}
  \bibitem{catalan} Catalan's Constant (最終閲覧日:2021/01/03)
  \\\url{https://mathworld.wolfram.com/CatalansConstant.html}
\end{thebibliography}






\end{document}